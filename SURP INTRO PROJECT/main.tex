\documentclass{article}
\usepackage[utf8]{inputenc}

\title{SURP 2021 Intro Project CTA200H}
\author{Niharika Namulla}
\date{May 2021}

\usepackage[utf8]{inputenc}
\usepackage{graphicx}

\setlength{\parindent}{4em}
\setlength{\parskip}{1em}

\begin{document}

\maketitle

\section{Introduction to Rebound}

In this section, REBOUND, a software package that integrates the motion of particles under the influence of gravity is used to simulate the evolution of a particle(planet) and the binary star system that it is bound by.
User inputted values include:
\\ $>$ For the binary system: mass ratio $mu$ = $m2/(m1+m2)$ = 0.5, semi-major axis ($a_b$ = 1), eccentricity ($e_b$ = 0.5), 
\\ $>$ For the orbiting particle: semi-major axis($a_p$ = 4; with respect to the center of mass), eccentricity ($e_p$ = 0)
\\ The binary system and the planet were initialized with random true anomaly. A flag is included in the code such that the simulation ends if the "planet" flies beyond the user-specified multiple of ab (in this case maximum distance = 100*$a_b$) from the centre of mass. 
The simulation is run for 10 orbital periods, and the position of each star along with the planet are plotted as parameterized curves x(t), y(t) for each orbiting body.
The results from this can be seen from Figure 1. 
\begin{figure}[htbp]
\includegraphics{Sec2_Bin_Sim.png}
\caption{'Binary system and test particle simulation using REBOUND. ($mu$ = 0.5, $a_b$ = 1, $e_b$ = 0.5, $a_p$ = 4, $e_p$ = 0)'}
\end{figure}

A plot of the simulation was also made using REBOUND's OrbitPlot function. This function plots the instantaneous orbits of binary system in the xy-plane (Figure 2.).

\begin{figure}[htbp]
\includegraphics{S2Orb_Plt.png}
\caption{'Binary system and test particle plot using REBOUND's OrbitPlot function '}
\end{figure}

\\  Comparing Figure 1. and 2. we see that while in Figure 1. we see 3 orbits, one for each body, in Figure 2. we see only 2 orbits. This is because orbits are defined between a pair or particles and not for each particle individually.

\\ Setting $a_p/a_b$ = 3, we obtain Figure 3. and Figure 4. All other parameters being the same, we notice that the the test particle shows pericenter precession.

\begin{figure}[htbp]
\includegraphics{S2_changea.png}
\caption{'Binary system and test particle (when $a_p/a_b = 3$)'}
\end{figure}

\begin{figure}[htbp]
\includegraphics{OrbPlt_changea.png}
\caption{'Binary system/planet (when $a_p/a_b = 3$) using REBOUND's OrbitPlot function '}
\end{figure}

\newpage

\section{CLASSIC RESULTS}

In this section we reproduce the stability criterion found by Holman \& Wiergert (1999) for 'P-type' orbits. The following equations provide the critical semi-major axis ( $a_c$ ) with respect to $e_b$ (where $a_c$ is observed to be independent of the binary mass ratio).

\begin{equation}
\displaystyle
\\a_c = 1.6.+ 5.1e-2.22e^2 + 4.12mu-.27e*mu-5.09mu^2 + 4.61e^2mu^2
\end{equation}

\begin{equation}
\displaystyle
\\a_c = a_b \approx 2.278 + 3.824e - 1.71e^2
\end{equation}

We start by running a suite of simulations for fixed $mu$ = 0.5, $e_p$ =0, $a_b$ = 1 values but varying $e_b$ over [0,0.7] and $a_p/a_b$ over [1,5]. We integrate for the particles position for a maximum of $10^4$ orbital periods and find the survival times for each input of ($e_b$,$a_p$) pair. This is done with the help of rebound.InterruptiblePool class. If the particle remains bound to the binary system till the end of the simulation, we simply take the end time at $10^4$ orbital periods to be the survival time.
Furthermore, the time required to run this simulation was obtained from Python's 'time module'. Running the simulation for 5 values of both $e_b$ and $a_p$ results in a total of 25 pairs of ($e_b$,$a_p$) values. For a total of 25 runs, each with a maximum of $10^4$ orbital periods it took  the simulation a total of 9460.94 seconds to complete. This simulation was run on a system with 2 cores and 4 logical processors, the results of which were plotted in Figure 5. 
\\
I realized that there was an issue in the code, and so I received the same value for each survival time. However due to time constraints I was not able to obtain the results of the updated code. What we should be seeing is a mesh of various colors that correspond to the survival times seen on the colorbar in the figure.

\begin{figure}[htbp]
\includegraphics{Pool.png}
\caption{Color-map giving the survival times of test particle for a series of $a_p$ and $e_b$ values'}
\end{figure}

However, a plot of varying $e_b$ values with respect to $a_b$ values was made and can be seen in Figure 6. Comparing this to Figure 4. of Holman & Wiegert (1999) we see that we received similar results. Similarly running the simulations for $10^4$ orbitals we see that there is an erosion of the stable outer region over time, especially at the larger eccentricities. Our plot is somewhat linear for the smaller eccentricities and nonlinear at higher eccentricities. 


\begin{figure}[htbp]
\includegraphics{ac_vs_eb.png}
\caption{'In the outer P-type region '}
\end{figure}






\end{document}
